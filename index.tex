% Options for packages loaded elsewhere
\PassOptionsToPackage{unicode}{hyperref}
\PassOptionsToPackage{hyphens}{url}
\PassOptionsToPackage{dvipsnames,svgnames,x11names}{xcolor}
%
\documentclass[
  letterpaper,
  DIV=11,
  numbers=noendperiod]{scrreprt}

\usepackage{amsmath,amssymb}
\usepackage{iftex}
\ifPDFTeX
  \usepackage[T1]{fontenc}
  \usepackage[utf8]{inputenc}
  \usepackage{textcomp} % provide euro and other symbols
\else % if luatex or xetex
  \usepackage{unicode-math}
  \defaultfontfeatures{Scale=MatchLowercase}
  \defaultfontfeatures[\rmfamily]{Ligatures=TeX,Scale=1}
\fi
\usepackage{lmodern}
\ifPDFTeX\else  
    % xetex/luatex font selection
\fi
% Use upquote if available, for straight quotes in verbatim environments
\IfFileExists{upquote.sty}{\usepackage{upquote}}{}
\IfFileExists{microtype.sty}{% use microtype if available
  \usepackage[]{microtype}
  \UseMicrotypeSet[protrusion]{basicmath} % disable protrusion for tt fonts
}{}
\makeatletter
\@ifundefined{KOMAClassName}{% if non-KOMA class
  \IfFileExists{parskip.sty}{%
    \usepackage{parskip}
  }{% else
    \setlength{\parindent}{0pt}
    \setlength{\parskip}{6pt plus 2pt minus 1pt}}
}{% if KOMA class
  \KOMAoptions{parskip=half}}
\makeatother
\usepackage{xcolor}
\setlength{\emergencystretch}{3em} % prevent overfull lines
\setcounter{secnumdepth}{5}
% Make \paragraph and \subparagraph free-standing
\ifx\paragraph\undefined\else
  \let\oldparagraph\paragraph
  \renewcommand{\paragraph}[1]{\oldparagraph{#1}\mbox{}}
\fi
\ifx\subparagraph\undefined\else
  \let\oldsubparagraph\subparagraph
  \renewcommand{\subparagraph}[1]{\oldsubparagraph{#1}\mbox{}}
\fi

\usepackage{color}
\usepackage{fancyvrb}
\newcommand{\VerbBar}{|}
\newcommand{\VERB}{\Verb[commandchars=\\\{\}]}
\DefineVerbatimEnvironment{Highlighting}{Verbatim}{commandchars=\\\{\}}
% Add ',fontsize=\small' for more characters per line
\usepackage{framed}
\definecolor{shadecolor}{RGB}{241,243,245}
\newenvironment{Shaded}{\begin{snugshade}}{\end{snugshade}}
\newcommand{\AlertTok}[1]{\textcolor[rgb]{0.68,0.00,0.00}{#1}}
\newcommand{\AnnotationTok}[1]{\textcolor[rgb]{0.37,0.37,0.37}{#1}}
\newcommand{\AttributeTok}[1]{\textcolor[rgb]{0.40,0.45,0.13}{#1}}
\newcommand{\BaseNTok}[1]{\textcolor[rgb]{0.68,0.00,0.00}{#1}}
\newcommand{\BuiltInTok}[1]{\textcolor[rgb]{0.00,0.23,0.31}{#1}}
\newcommand{\CharTok}[1]{\textcolor[rgb]{0.13,0.47,0.30}{#1}}
\newcommand{\CommentTok}[1]{\textcolor[rgb]{0.37,0.37,0.37}{#1}}
\newcommand{\CommentVarTok}[1]{\textcolor[rgb]{0.37,0.37,0.37}{\textit{#1}}}
\newcommand{\ConstantTok}[1]{\textcolor[rgb]{0.56,0.35,0.01}{#1}}
\newcommand{\ControlFlowTok}[1]{\textcolor[rgb]{0.00,0.23,0.31}{#1}}
\newcommand{\DataTypeTok}[1]{\textcolor[rgb]{0.68,0.00,0.00}{#1}}
\newcommand{\DecValTok}[1]{\textcolor[rgb]{0.68,0.00,0.00}{#1}}
\newcommand{\DocumentationTok}[1]{\textcolor[rgb]{0.37,0.37,0.37}{\textit{#1}}}
\newcommand{\ErrorTok}[1]{\textcolor[rgb]{0.68,0.00,0.00}{#1}}
\newcommand{\ExtensionTok}[1]{\textcolor[rgb]{0.00,0.23,0.31}{#1}}
\newcommand{\FloatTok}[1]{\textcolor[rgb]{0.68,0.00,0.00}{#1}}
\newcommand{\FunctionTok}[1]{\textcolor[rgb]{0.28,0.35,0.67}{#1}}
\newcommand{\ImportTok}[1]{\textcolor[rgb]{0.00,0.46,0.62}{#1}}
\newcommand{\InformationTok}[1]{\textcolor[rgb]{0.37,0.37,0.37}{#1}}
\newcommand{\KeywordTok}[1]{\textcolor[rgb]{0.00,0.23,0.31}{#1}}
\newcommand{\NormalTok}[1]{\textcolor[rgb]{0.00,0.23,0.31}{#1}}
\newcommand{\OperatorTok}[1]{\textcolor[rgb]{0.37,0.37,0.37}{#1}}
\newcommand{\OtherTok}[1]{\textcolor[rgb]{0.00,0.23,0.31}{#1}}
\newcommand{\PreprocessorTok}[1]{\textcolor[rgb]{0.68,0.00,0.00}{#1}}
\newcommand{\RegionMarkerTok}[1]{\textcolor[rgb]{0.00,0.23,0.31}{#1}}
\newcommand{\SpecialCharTok}[1]{\textcolor[rgb]{0.37,0.37,0.37}{#1}}
\newcommand{\SpecialStringTok}[1]{\textcolor[rgb]{0.13,0.47,0.30}{#1}}
\newcommand{\StringTok}[1]{\textcolor[rgb]{0.13,0.47,0.30}{#1}}
\newcommand{\VariableTok}[1]{\textcolor[rgb]{0.07,0.07,0.07}{#1}}
\newcommand{\VerbatimStringTok}[1]{\textcolor[rgb]{0.13,0.47,0.30}{#1}}
\newcommand{\WarningTok}[1]{\textcolor[rgb]{0.37,0.37,0.37}{\textit{#1}}}

\providecommand{\tightlist}{%
  \setlength{\itemsep}{0pt}\setlength{\parskip}{0pt}}\usepackage{longtable,booktabs,array}
\usepackage{calc} % for calculating minipage widths
% Correct order of tables after \paragraph or \subparagraph
\usepackage{etoolbox}
\makeatletter
\patchcmd\longtable{\par}{\if@noskipsec\mbox{}\fi\par}{}{}
\makeatother
% Allow footnotes in longtable head/foot
\IfFileExists{footnotehyper.sty}{\usepackage{footnotehyper}}{\usepackage{footnote}}
\makesavenoteenv{longtable}
\usepackage{graphicx}
\makeatletter
\def\maxwidth{\ifdim\Gin@nat@width>\linewidth\linewidth\else\Gin@nat@width\fi}
\def\maxheight{\ifdim\Gin@nat@height>\textheight\textheight\else\Gin@nat@height\fi}
\makeatother
% Scale images if necessary, so that they will not overflow the page
% margins by default, and it is still possible to overwrite the defaults
% using explicit options in \includegraphics[width, height, ...]{}
\setkeys{Gin}{width=\maxwidth,height=\maxheight,keepaspectratio}
% Set default figure placement to htbp
\makeatletter
\def\fps@figure{htbp}
\makeatother
% definitions for citeproc citations
\NewDocumentCommand\citeproctext{}{}
\NewDocumentCommand\citeproc{mm}{%
  \begingroup\def\citeproctext{#2}\cite{#1}\endgroup}
\makeatletter
 % allow citations to break across lines
 \let\@cite@ofmt\@firstofone
 % avoid brackets around text for \cite:
 \def\@biblabel#1{}
 \def\@cite#1#2{{#1\if@tempswa , #2\fi}}
\makeatother
\newlength{\cslhangindent}
\setlength{\cslhangindent}{1.5em}
\newlength{\csllabelwidth}
\setlength{\csllabelwidth}{3em}
\newenvironment{CSLReferences}[2] % #1 hanging-indent, #2 entry-spacing
 {\begin{list}{}{%
  \setlength{\itemindent}{0pt}
  \setlength{\leftmargin}{0pt}
  \setlength{\parsep}{0pt}
  % turn on hanging indent if param 1 is 1
  \ifodd #1
   \setlength{\leftmargin}{\cslhangindent}
   \setlength{\itemindent}{-1\cslhangindent}
  \fi
  % set entry spacing
  \setlength{\itemsep}{#2\baselineskip}}}
 {\end{list}}
\usepackage{calc}
\newcommand{\CSLBlock}[1]{\hfill\break\parbox[t]{\linewidth}{\strut\ignorespaces#1\strut}}
\newcommand{\CSLLeftMargin}[1]{\parbox[t]{\csllabelwidth}{\strut#1\strut}}
\newcommand{\CSLRightInline}[1]{\parbox[t]{\linewidth - \csllabelwidth}{\strut#1\strut}}
\newcommand{\CSLIndent}[1]{\hspace{\cslhangindent}#1}

\KOMAoption{captions}{tableheading}
\makeatletter
\@ifpackageloaded{tcolorbox}{}{\usepackage[skins,breakable]{tcolorbox}}
\@ifpackageloaded{fontawesome5}{}{\usepackage{fontawesome5}}
\definecolor{quarto-callout-color}{HTML}{909090}
\definecolor{quarto-callout-note-color}{HTML}{0758E5}
\definecolor{quarto-callout-important-color}{HTML}{CC1914}
\definecolor{quarto-callout-warning-color}{HTML}{EB9113}
\definecolor{quarto-callout-tip-color}{HTML}{00A047}
\definecolor{quarto-callout-caution-color}{HTML}{FC5300}
\definecolor{quarto-callout-color-frame}{HTML}{acacac}
\definecolor{quarto-callout-note-color-frame}{HTML}{4582ec}
\definecolor{quarto-callout-important-color-frame}{HTML}{d9534f}
\definecolor{quarto-callout-warning-color-frame}{HTML}{f0ad4e}
\definecolor{quarto-callout-tip-color-frame}{HTML}{02b875}
\definecolor{quarto-callout-caution-color-frame}{HTML}{fd7e14}
\makeatother
\makeatletter
\@ifpackageloaded{bookmark}{}{\usepackage{bookmark}}
\makeatother
\makeatletter
\@ifpackageloaded{caption}{}{\usepackage{caption}}
\AtBeginDocument{%
\ifdefined\contentsname
  \renewcommand*\contentsname{Table of contents}
\else
  \newcommand\contentsname{Table of contents}
\fi
\ifdefined\listfigurename
  \renewcommand*\listfigurename{List of Figures}
\else
  \newcommand\listfigurename{List of Figures}
\fi
\ifdefined\listtablename
  \renewcommand*\listtablename{List of Tables}
\else
  \newcommand\listtablename{List of Tables}
\fi
\ifdefined\figurename
  \renewcommand*\figurename{Figure}
\else
  \newcommand\figurename{Figure}
\fi
\ifdefined\tablename
  \renewcommand*\tablename{Table}
\else
  \newcommand\tablename{Table}
\fi
}
\@ifpackageloaded{float}{}{\usepackage{float}}
\floatstyle{ruled}
\@ifundefined{c@chapter}{\newfloat{codelisting}{h}{lop}}{\newfloat{codelisting}{h}{lop}[chapter]}
\floatname{codelisting}{Listing}
\newcommand*\listoflistings{\listof{codelisting}{List of Listings}}
\makeatother
\makeatletter
\makeatother
\makeatletter
\@ifpackageloaded{caption}{}{\usepackage{caption}}
\@ifpackageloaded{subcaption}{}{\usepackage{subcaption}}
\makeatother
\makeatletter
\@ifpackageloaded{tikz}{}{\usepackage{tikz}}
\makeatother
        \newcommand*\circled[1]{\tikz[baseline=(char.base)]{
          \node[shape=circle,draw,inner sep=1pt] (char) {{\scriptsize#1}};}}  
                  
\ifLuaTeX
  \usepackage{selnolig}  % disable illegal ligatures
\fi
\usepackage{bookmark}

\IfFileExists{xurl.sty}{\usepackage{xurl}}{} % add URL line breaks if available
\urlstyle{same} % disable monospaced font for URLs
\hypersetup{
  pdftitle={Oleson Lab Reseach Project Resource },
  pdfauthor={Alemarie Ceria},
  colorlinks=true,
  linkcolor={blue},
  filecolor={Maroon},
  citecolor={Blue},
  urlcolor={Blue},
  pdfcreator={LaTeX via pandoc}}

\title{Oleson Lab Reseach Project Resource}
\author{Alemarie Ceria}
\date{2024-01-30}

\begin{document}
\maketitle

\renewcommand*\contentsname{Table of contents}
{
\hypersetup{linkcolor=}
\setcounter{tocdepth}{2}
\tableofcontents
}
\bookmarksetup{startatroot}

\chapter*{Overview}\label{overview}
\addcontentsline{toc}{chapter}{Overview}

\markboth{Overview}{Overview}

This resource seeks to provide you with the following:

\begin{tcolorbox}[enhanced jigsaw, colback=white, left=2mm, colframe=quarto-callout-color-frame, opacityback=0, leftrule=.75mm, breakable, arc=.35mm, rightrule=.15mm, bottomrule=.15mm, toprule=.15mm]

\vspace{-3mm}\textbf{Interactive Code Annotations}\vspace{3mm}

Throughout this resource, hover over the numbered annotations to the
right of code chunks to reveal detailed explanations and comments about
the code.

\end{tcolorbox}

\bookmarksetup{startatroot}

\chapter*{Software and Platforms}\label{sec-software-and-platforms}
\addcontentsline{toc}{chapter}{Software and Platforms}

\markboth{Software and Platforms}{Software and Platforms}

\section*{Download/Sign Up Links}\label{sec-download-sign-up-links}
\addcontentsline{toc}{section}{Download/Sign Up Links}

\markright{Download/Sign Up Links}

\begin{longtable}[]{@{}
  >{\raggedright\arraybackslash}p{(\columnwidth - 4\tabcolsep) * \real{0.1227}}
  >{\raggedright\arraybackslash}p{(\columnwidth - 4\tabcolsep) * \real{0.3727}}
  >{\raggedright\arraybackslash}p{(\columnwidth - 4\tabcolsep) * \real{0.5000}}@{}}
\toprule\noalign{}
\begin{minipage}[b]{\linewidth}\raggedright
Software/Platform
\end{minipage} & \begin{minipage}[b]{\linewidth}\raggedright
Purpose
\end{minipage} & \begin{minipage}[b]{\linewidth}\raggedright
Download/Sign Up Link
\end{minipage} \\
\midrule\noalign{}
\endhead
\bottomrule\noalign{}
\endlastfoot
Slack & Enhances team communication and project coordination &
\href{https://slack.com/intl/en-in/downloads/windows}{Windows Link}

\href{https://slack.com/intl/en-in/downloads/mac}{Mac Link} \\
Google Drive for Desktop & Streamlines file storage, sharing, and
collaboration & \href{https://www.google.com/drive/download/}{Link} \\
R & Provides tools for statistical analysis and data visualization &
\href{https://cran.rstudio.com/}{Link} \\
RStudio & Facilitates R coding, debugging, and project organization &
\href{https://posit.co/download/rstudio-desktop/}{Link} \\
Git & Manages version control for tracking and merging changes in code
and documents & \href{https://git-scm.com/downloads}{Link} \\
GitHub & Hosts and manages Git repositories, facilitating collaboration
and code sharing &
\href{https://github.com/signup?ref_cta=Sign+up&ref_loc=header+logged+out&ref_page=\%2F&source=header-home}{Link} \\
Zotero & Organizes and cites research sources consistently &
\href{https://www.zotero.org/download/}{Link} \\
\end{longtable}

\section*{Setup Instructions}\label{setup-instructions}
\addcontentsline{toc}{section}{Setup Instructions}

\markright{Setup Instructions}

\bookmarksetup{startatroot}

\chapter*{Project Organization}\label{project-organization}
\addcontentsline{toc}{chapter}{Project Organization}

\markboth{Project Organization}{Project Organization}

\section*{Naming Conventions}\label{naming-conventions}
\addcontentsline{toc}{section}{Naming Conventions}

\markright{Naming Conventions}

\url{https://www.youtube.com/watch?v=ES1LTlnpLMk&list=PLy_EwS4oOnoQzWYaWDO_x5t7O1RqHwuhw&index=5&pp=gAQBiAQB}

\textbf{TL-DR}:

Best practices when naming directories and files:

\begin{enumerate}
\def\labelenumi{\arabic{enumi}.}
\tightlist
\item
  Make it machine readable

  \begin{itemize}
  \tightlist
  \item
    Use globbing to isolate and find files that match a simple pattern
    using regular expressions in a scripting language

    \begin{itemize}
    \tightlist
    \item
      Eg., The following bash code will find all file names that start
      with \texttt{report-2020-} and ends with \texttt{.txt}
      \texttt{report-2020-*.txt}
    \end{itemize}
  \item
    No spaces or accented characters
  \item
    No punctuation other than hyphens (-) and underscores (\_)
  \item
    Use lowercase only
  \end{itemize}
\item
  Make it human readable

  \begin{itemize}
  \tightlist
  \item
    Use highly informative slugs (e.g.,
    \texttt{01\_collect-cdec-snow-data.R})

    \begin{itemize}
    \tightlist
    \item
      Should be descriptive and should accurately reflect the content
    \item
      Should be concise
    \item
      Use plain language, but still readable and understandable
    \item
      Use hyphenation for spaces
    \item
      Use underscores to separate metadata that needs to be parsed or
      processed in a scripting language using regular expressions
    \item
      Avoid special characters
    \end{itemize}
  \end{itemize}
\item
  Sort it in a useful way

  \begin{itemize}
  \tightlist
  \item
    Put something numeric first
  \item
    Logical sorting

    \begin{itemize}
    \tightlist
    \item
      Apply left padding (e.g., 01, 02, etc.) to single digit numbers in
      file names to maintain their sequential order
    \item
      Use when you have a predefined sequence or steps in a process
      (e.g., stages in a workflow)
    \end{itemize}
  \item
    Chronological sorting

    \begin{itemize}
    \tightlist
    \item
      Follow the
      \href{https://www.iso.org/iso-8601-date-and-time-format.html}{ISO
      8601} standard when formatting dates: YYYY-MM-DD (e.g.,
      \texttt{2020-12-01\_draft.docx})
    \item
      Use when the timeline of creation, modification, or relevance is
      important (e.g., In projects with regular updates or when you need
      to track progress or changes over time where the most recent date
      indicates the latest version)
    \end{itemize}
  \end{itemize}
\end{enumerate}

\begin{longtable}[]{@{}lll@{}}
\toprule\noalign{}
Type & Naming Convention & Example \\
\midrule\noalign{}
\endhead
\bottomrule\noalign{}
\endlastfoot
& & \\
& & \\
& & \\
\end{longtable}

\section*{Directories}\label{directories}
\addcontentsline{toc}{section}{Directories}

\markright{Directories}

\subsection*{Local - Shared Google
Drive}\label{local---shared-google-drive}
\addcontentsline{toc}{subsection}{Local - Shared Google Drive}

\subsubsection*{Example Project Directory
Structure}\label{example-project-directory-structure}
\addcontentsline{toc}{subsubsection}{Example Project Directory
Structure}

\phantomsection\label{annotated-cell-2}%
\begin{Shaded}
\begin{Highlighting}[]
\NormalTok{📦}\OperatorTok{/}\NormalTok{g}\OperatorTok{/}\NormalTok{path}\OperatorTok{/}\NormalTok{to}\OperatorTok{/}\NormalTok{example}\OperatorTok{{-}}\NormalTok{project}\OperatorTok{{-}}\NormalTok{directory }\hspace*{\fill}\NormalTok{\circled{1}}
\NormalTok{ ┣ 📂deliverables }\hspace*{\fill}\NormalTok{\circled{2}}
\NormalTok{ ┃ ┗ 📂papers }\hspace*{\fill}\NormalTok{\circled{3}}
\NormalTok{ ┣ 📂github}\OperatorTok{{-}}\NormalTok{repo }\hspace*{\fill}\NormalTok{\circled{4}}
\NormalTok{ ┃ ┗ 📂example}\OperatorTok{{-}}\NormalTok{project}\OperatorTok{{-}}\NormalTok{github}\OperatorTok{{-}}\NormalTok{repo }
\NormalTok{ ┣ 📂meetings}\OperatorTok{{-}}\KeywordTok{and}\OperatorTok{{-}}\NormalTok{events }\hspace*{\fill}\NormalTok{\circled{5}}
\NormalTok{ ┃ ┣ 📂meeting}\OperatorTok{{-}}\NormalTok{notes }\hspace*{\fill}\NormalTok{\circled{6}}
\NormalTok{ ┃ ┣ 📂agendas }\hspace*{\fill}\NormalTok{\circled{7}}
\NormalTok{ ┃ ┗ 📂workshop}\OperatorTok{{-}}\KeywordTok{or}\OperatorTok{{-}}\NormalTok{event}\OperatorTok{{-}}\NormalTok{planning }\hspace*{\fill}\NormalTok{\circled{8}}
\NormalTok{ ┣ 📂presentations }\hspace*{\fill}\NormalTok{\circled{9}}
\NormalTok{ ┣ 📂project}\OperatorTok{{-}}\NormalTok{materials }\hspace*{\fill}\NormalTok{\circled{10}}
\NormalTok{ ┃ ┣ 📂methods}\OperatorTok{{-}}\NormalTok{drafts }\hspace*{\fill}\NormalTok{\circled{11}}
\NormalTok{ ┃ ┗ 📂lit}\OperatorTok{{-}}\NormalTok{review }\hspace*{\fill}\NormalTok{\circled{12}}
\NormalTok{ ┗ 📜README.md }\hspace*{\fill}\NormalTok{\circled{13}}
\end{Highlighting}
\end{Shaded}

\begin{description}
\tightlist
\item[\circled{1}]
Root directory housing all materials and documentation for the research
project.
\item[\circled{2}]
Contains final outputs like reports, papers, or products resulting from
the project.
\item[\circled{3}]
Stores final versions and supplementary materials for academic papers or
reports.
\item[\circled{4}]
Specific GitHub repository directory for collaborative development and
code sharing. \emph{Only \texttt{example-project-github-repo/} will be
pushed to GitHub.}
\item[\circled{5}]
Organizes documentation and planning materials for meetings and events
related to the project.
\item[\circled{6}]
Archives notes and decisions from project meetings.
\item[\circled{7}]
Prepares and stores agendas for upcoming meetings to structure
discussions.
\item[\circled{8}]
Holds planning documents, schedules, and resources for workshops or
project-related events.
\item[\circled{9}]
Contains slides, speaker notes, and related materials for presentations
about the project.
\item[\circled{10}]
Stores various project-related documents not categorized elsewhere.
\item[\circled{11}]
Keeps drafts and notes on methodological approaches and procedures.
\item[\circled{12}]
Compiles literature reviews, reference materials, and bibliographies.
\item[\circled{13}]
Provides an overview of the project directory, explaining the structure
and contents of the folders/files.
\end{description}

\subsubsection*{Create New Project
Directory}\label{create-new-project-directory}
\addcontentsline{toc}{subsubsection}{Create New Project Directory}

The following code creates a directory for your project within the
\texttt{current-projects/} directory in the \texttt{Oleson\ Lab/} Shared
Google Drive by making a copy of an existing template.

To execute this code chunk, please do the following:

\begin{enumerate}
\def\labelenumi{\arabic{enumi}.}
\tightlist
\item
  Ensure you have \href{https://www.google.com/drive/download/}{Google
  Drive for Desktop} software downloaded, you are signed in, and are
  able to see the \texttt{Oleson\ Lab/} directory in your
  \texttt{Shared\ Drives/} directory
\item
  Change \texttt{\#\textbar{}\ eval:\ false} to
  \texttt{\#\textbar{}\ eval:\ true}
\item
  Input your desired project name in the code
  \texttt{project\_name="example-project"}.
\end{enumerate}

\phantomsection\label{annotated-cell-3}%
\begin{Shaded}
\begin{Highlighting}[]
\InformationTok{\textasciigrave{}\textasciigrave{}\textasciigrave{}\{bash\}}
\InformationTok{\#| label: create{-}new{-}project{-}directory}
\InformationTok{\#| code{-}overflow: wrap}
\InformationTok{\#| eval: false \# \textless{}1\textgreater{}}

\InformationTok{echo "Starting script..." \# \textless{}2\textgreater{}}

\InformationTok{cd "/g/Shared drives/Oleson Lab/projects/project{-}management{-}resources" \# \textless{}3\textgreater{}}

\InformationTok{new\_project\_template="/g/Shared drives/Oleson Lab/projects/project{-}management{-}resources/new{-}project{-}template" \# \textless{}4\textgreater{}}

\InformationTok{project\_name="example{-}project" \# \textless{}4\textgreater{}}

\InformationTok{new\_current\_project\_directory="/g/Shared drives/Oleson Lab/projects/current{-}projects/$project\_name" \# \textless{}5\textgreater{}}

\InformationTok{\# Check if the directory already exists}
\InformationTok{if [ {-}d "$new\_current\_project\_directory" ]; then \# \textless{}6\textgreater{}}
\InformationTok{    echo "Error: A project with the name \textquotesingle{}$project\_name\textquotesingle{} already exists. Please choose a different name." \# \textless{}6\textgreater{}}
\InformationTok{    exit 1 \# \textless{}6\textgreater{}}
\InformationTok{fi \# \textless{}6\textgreater{}}

\InformationTok{\# Copy the new{-}project{-}template directory to the new project directory}
\InformationTok{cp {-}r "$new\_project\_template" "$new\_current\_project\_directory" \# \textless{}7\textgreater{}}

\InformationTok{echo "Completed copying." \# \textless{}8\textgreater{}}
\InformationTok{\textasciigrave{}\textasciigrave{}\textasciigrave{}}
\end{Highlighting}
\end{Shaded}

\begin{description}
\tightlist
\item[\circled{1}]
\textbf{To run code chunk, set \texttt{false} to \texttt{true}.}
\item[\circled{2}]
Confirms that the script is initiated.
\item[\circled{3}]
Navigates to directory containing the new project template.
\item[\circled{4}]
Defines a variable \texttt{new\_project\_template} which stores the
\texttt{new-project-template/} directory file path.
\item[\circled{5}]
Sets project\_name to desired project name and constructs a path for the
new project in the \texttt{current-projects} directory. \textbf{Replace
\texttt{{[}insert-project-name{]}}.}
\item[\circled{6}]
Checks to see whether or not the directory exists within the
\texttt{current-projects/} directory and if so, prompts you to choose
another name.
\item[\circled{7}]
Copies everything within the \texttt{new-project-template/}directory
into the \texttt{current-projects/} directory and renames it
accordingly.
\item[\circled{8}]
Confirms that the script ran successfully.
\end{description}

\subsection*{Remote - GitHub
Repository}\label{remote---github-repository}
\addcontentsline{toc}{subsection}{Remote - GitHub Repository}

\subsubsection*{Example GitHub Repository
Structure}\label{example-github-repository-structure}
\addcontentsline{toc}{subsubsection}{Example GitHub Repository
Structure}

\phantomsection\label{annotated-cell-4}%
\begin{Shaded}
\begin{Highlighting}[]
\NormalTok{📦}\OperatorTok{/}\NormalTok{g}\OperatorTok{/}\NormalTok{path}\OperatorTok{/}\NormalTok{to}\OperatorTok{/}\NormalTok{example}\OperatorTok{{-}}\NormalTok{project }\hspace*{\fill}\NormalTok{\circled{1}}
\NormalTok{ ┣ 📂outputs }\hspace*{\fill}\NormalTok{\circled{2}}
\NormalTok{ ┃ ┣ 📂tables }\hspace*{\fill}\NormalTok{\circled{3}}
\NormalTok{ ┃ ┗ 📂plots }\hspace*{\fill}\NormalTok{\circled{4}}
\NormalTok{ ┣ 📂documentation }\hspace*{\fill}\NormalTok{\circled{5}}
\NormalTok{ ┣ 📂code }\hspace*{\fill}\NormalTok{\circled{6}}
\NormalTok{ ┃ ┣ 📂functions }\hspace*{\fill}\NormalTok{\circled{7}}
\NormalTok{ ┃ ┗ 📂models }\hspace*{\fill}\NormalTok{\circled{8}}
\NormalTok{ ┣ 📂data }\hspace*{\fill}\NormalTok{\circled{9}}
\NormalTok{ ┃ ┣ 📂exploratory }\hspace*{\fill}\NormalTok{\circled{10}}
\NormalTok{ ┃ ┣ 📂raw }\hspace*{\fill}\NormalTok{\circled{11}}
\NormalTok{ ┃ ┣ 📂tidied }\hspace*{\fill}\NormalTok{\circled{12}}
\NormalTok{ ┃ ┗ 📂output }\hspace*{\fill}\NormalTok{\circled{13}}
\NormalTok{ ┣ 📜LICENSE }\hspace*{\fill}\NormalTok{\circled{14}}
\NormalTok{ ┣ 📜.gitignore }\hspace*{\fill}\NormalTok{\circled{15}}
\NormalTok{ ┣ 📜example}\OperatorTok{{-}}\NormalTok{project.Rproj }\hspace*{\fill}\NormalTok{\circled{16}}
\NormalTok{ ┣ 📜README.qmd }\hspace*{\fill}\NormalTok{\circled{17}}
\NormalTok{ ┗ 📜README.md }\hspace*{\fill}\NormalTok{\circled{18}}
\end{Highlighting}
\end{Shaded}

\begin{description}
\tightlist
\item[\circled{1}]
Serves as the root directory, encapsulating all project components for
easy management and navigation.
\item[\circled{2}]
Contains generated files like tables and plots, separating results from
the input data and code.
\item[\circled{3}]
Stores tabular results and data summaries.
\item[\circled{4}]
Holds visualizations and graphs generated by the analysis.
\item[\circled{5}]
Stores project documentation, reports, and notes, centralizing
information for reference and clarity.
\item[\circled{6}]
Contains all the scripts and code used in the project, promoting
modularity and code reuse.
\item[\circled{7}]
Stores custom functions to ensure code modularity and readability.
\item[\circled{8}]
Holds existing and output models.
\item[\circled{9}]
Acts as the main repository for all datasets, organized to reflect
different stages of data processing.
\item[\circled{10}]
Contains initial explorations and analyses, fostering a sandbox
environment for preliminary insights. (Will not be pushed to GitHub)
\item[\circled{11}]
Stores unmodified raw data, preserving the original datasets for
reproducibility and reference.
\item[\circled{12}]
Holds processed and cleaned data, ready for analysis, ensuring
consistency and reliability. (Will not be pushed to GitHub)
\item[\circled{13}]
Stores the final dataset used for analysis.
\item[\circled{14}]
Holds processed and cleaned data, ready for analysis, ensuring
consistency and reliability.
\item[\circled{15}]
Specifies the terms under which the project can be used or distributed,
clarifying legal and usage aspects.
\item[\circled{16}]
Lists files and directories to be ignored by version control, keeping
the repository clean and relevant.
\item[\circled{17}]
Provides a Quarto-rendered, detailed project overview and instructions,
enhancing comprehension and usage.
\item[\circled{18}]
Output of \texttt{README.qmd} used to be displayed on GitHub Repository.
\end{description}

\bookmarksetup{startatroot}

\chapter*{Data}\label{sec-data}
\addcontentsline{toc}{chapter}{Data}

\markboth{Data}{Data}

\bookmarksetup{startatroot}

\chapter*{Code}\label{sec-code}
\addcontentsline{toc}{chapter}{Code}

\markboth{Code}{Code}

\bookmarksetup{startatroot}

\chapter*{Reports and Publications}\label{sec-reports-and-publications}
\addcontentsline{toc}{chapter}{Reports and Publications}

\markboth{Reports and Publications}{Reports and Publications}

\bookmarksetup{startatroot}

\chapter*{More Resources}\label{sec-more-resources}
\addcontentsline{toc}{chapter}{More Resources}

\markboth{More Resources}{More Resources}

DMP and SOP:

\begin{itemize}
\tightlist
\item
  \href{https://github.com/WA-Department-of-Agriculture/washi-dmp}{WA-Department-of-Agriculture
  / washi-dmp}

  \begin{itemize}
  \tightlist
  \item
    \href{https://wa-department-of-agriculture.github.io/washi-dmp/}{Washington
    Soil Health Initiative: State of the Soils Assessment Data
    Management Plan}
  \end{itemize}
\item
  \href{https://emlab-ucsb.github.io/SOP/}{emLab Standard Operating
  Procedures}
\item
  \href{https://www.youtube.com/watch?v=m29u6OniOGQ&t=3446s}{Data
  Management for Scientists by David LeBauer}
\end{itemize}

Documentation:

\begin{itemize}
\tightlist
\item
  \href{https://documentation.divio.com/}{The Documentation System}

  \begin{itemize}
  \tightlist
  \item
    \href{https://www.youtube.com/watch?v=t4vKPhjcMZg}{What nobody tells
    you about documentation}
  \end{itemize}
\item
  \href{https://data.library.arizona.edu/data-management/best-practices/data-documentation-readme-metadata}{The
  University of Arizona Data Documentation Resource}\\
\item
  \href{https://github.com/genophenoenvo/neon-datasets/blob/main/README.md}{genophenoenvo/neon-datasets/README.md
  Example}
\end{itemize}

Videos:

\begin{itemize}
\tightlist
\item
  \href{https://www.youtube.com/playlist?list=PLy_EwS4oOnoQzWYaWDO_x5t7O1RqHwuhw}{Reproducible
  Research YouTube Playlist}
\end{itemize}

\bookmarksetup{startatroot}

\chapter*{References}\label{references}
\addcontentsline{toc}{chapter}{References}

\markboth{References}{References}

\phantomsection\label{refs}
\begin{CSLReferences}{0}{1}
\end{CSLReferences}



\end{document}
